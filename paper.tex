%-*- coding: UTF-8 -*-
% paper.tex
% 语言模型隐私风险与安全防护
\documentclass[12pt]{ctexart}
\usepackage{graphicx}
\usepackage{float}
\usepackage{cite}
\usepackage{geometry}
\usepackage{url}
\usepackage{fancyhdr}
\usepackage{appendix}
\usepackage{amsmath}
\usepackage{lipsum,mwe,cuted}
\usepackage{caption}
\usepackage{booktabs}
\usepackage{multirow}
\usepackage{amsfonts}

\setlength{\lineskip}{4em} 
\DeclareMathOperator*{\argmax}{arg\,max}
\DeclareMathOperator*{\argmin}{arg\,min}

\CTEXsetup[format={\Large\bfseries}]{section}

\pagestyle{plain}
% \geometry{a4paper,left=2cm,right=2cm,top=2cm,bottom=2cm}
\geometry{a4paper,scale=0.8}

\newcommand{\tabincell}[2]{\begin{tabular}{@{}#1@{}}#2\end{tabular}}

\newcommand{\upcite}[1]{\textsuperscript{\textsuperscript{\cite{#1}}}}

\title{\textbf{Principles of Research}}

\author{ 
	\textbf{by Albert Einstein 1918}
	\\ Physical Society, Berlin, for Max Planck’s sixtieth birtday
}


\bibliographystyle{unsrt}
\date{}

\begin{document}
\maketitle

\large{
In the temple of science are many mansions, and various indeed are they that dwell therein and the motives that have led them thither. Many take to science out of a joyful sense of superior intellectual power; science is their own special sport to which they look for vivid experience and the satisfaction of ambition; many others are to be found in the temple who have offered the products of their brains on this altar for purely utilitarian purposes. Were an angel of the Lord to come and drive all the people belonging to these two categories out of the temple, the assemblage would be seriously depleted, but there would still be some men, of both present and past times, left inside. Our Planck is one of them, and that is why we love him.



I am quite aware that we have just now lightheartedly expelled in imagination many excellent men who are largely, perhaps chiefly, responsible for the buildings of the temple of science; and in many cases our angel would find it a pretty ticklish job to decide. But of one thing I feel sure: if the types we have just expelled were the only types there were, the temple would never have come to be, any more than a forest can grow which consists of nothing but creepers. For these people any sphere of human activity will do, if it comes to a point; whether they become engineers, officers, tradesmen, or scientists depends on circumstances. Now let us have another look at those who have found favor with the angel. Most of them are somewhat odd, uncommunicative, solitary fellows, really less like each other, in spite of these common characteristics, than the hosts of the rejected. What has brought them to the temple? That is a difficult question and no single answer will cover it. To begin with, I believe with Schopenhauer that one of the strongest motives that leads men to art and science is escape from everyday life with its painful crudity and hopeless dreariness, from the fetters of one’s own ever shifting desires. A finely tempered nature longs to escape from personal life into the world of objective perception and thought; this desire may be compared with the townsman’s irresistible longing to escape from his noisy, cramped surroundings into the silence of high mountains, where the eye ranges freely through the still, pure air and fondly traces out the restful contours apparently built for eternity.


With this negative motive there goes a positive one. Man tries to make for himself in the fashion that suits him best a simplified and intelligible picture of the world; he then tries to some extent to substitute this cosmos of his for the world of experience, and thus to overcome it. This is what the painter, the poet, the speculative philosopher, and the natural scientist do, each in his own fashion. Each makes this cosmos and its construction the pivot of his emotional life, in order to find in this way the peace and security which he cannot find in tbe narrow whirlpool of personal experience.


What place does the theoretical physicist’s picture of the world occupy among all these possible pictures? It demands the highest possible standard of rigorous precision in the description of relations, such as only the use of mathematical language can give. In regard to his subject matter, on the other hand, the physicist has to limit himself very severely: he must content himself with describing the most simple events which can be brought within the domain of our experience; all events of a more complex order are beyond the power of the human intellect to reconstruct with the subtle accuracy and logical perfection which the theoretical physicist demands. Supreme purity, clarity, and certainty at the cost of completeness. But what can be the attraction of getting to know such a tiny section of nature thoroughly, while one leaves everything subtler and more complex shyly and timidly alone? Does the product of such a modest effort deserve to be called by the proud name of a theory of the universe?


In my belief the name is justified; for the general laws on which the structure of theoretical physics is based claim to be valid for any natural phenomenon whatsoever. With them, it ought to be possible to arrive at the description, that is to say, the theory, of every natural process, including life, by means of pure deduction, if that process of deduction were not far beyond the capacity of the human intellect. The physicist’s renunciation of completeness for his cosmos is therefore not a matter of fundamental principle.


The supreme task of the physicist is to arrive at those universal elementary laws from which the cosmos can be built up by pure deduction. There is no logical path to these laws; only intuition, resting on sympathetic understanding of experience, can reach them. In this methodological uncertainty, one might suppose that there were any number of possible systems of theoretical physics all equally well justified; and this opinion is no doubt correct, theoretically. But the development of physics has shown that at any given moment, out of all conceivable constructions, a single one has always proved itself decidedly superior to all the rest. Nobody who has really gone deeply into the matter will deny that in practice the world of phenomena uniquely determines the theoretical system, in spite of the fact that there is no logical bridge between phenomena and their theoretical principles; this is what Leibnitz described so happily as a “pre-established harmony.” Physicists often accuse epistemologists of not paying sufficient attention to this fact. Here, it seems to me, lie the roots of the controversy carried on some years ago between Mach and Planck.


The longing to behold this pre-established harmony is the source of the inexhaustible patience and perseverance with which Planck has devoted himself, as we see, to the most general problems of our science, refusing to let himself be diverted to more grateful and more easily attained ends. I have often heard colleagues try to attribute this attitude of his to extraordinary will-power and discipline — wrongly, in my opinion. The state of mind which enables a man to do work of this kind is akin to that of the religious worshiper or the lover; the daily effort comes from no deliberate intention or program, but straight from the heart. There he sits, our beloved Planck, and smiles inside himself at my childish playing-about with the lantern of Diogenes. Our affection for him needs no threadbare explanation. May the love of science continue to illumine his path in the future and lead him to the solution of the most important problem in present-day physics, which he has himself posed and done so much to solve. May he succeed in uniting quantum theory with electrodynamics and mechanics in a single logical system.
}

\newpage

\begin{center}
	\Huge{\textbf{科研的原则}}
	\linebreak
	\linebreak
	\large{\textbf{爱因斯坦 1918}}
	\linebreak
\end{center}


\noindent{\textbf{序言}:这是爱因斯坦于1918年4月在柏林物理学会举办的麦克斯·普朗克六十岁生日庆祝会上的讲话。讲稿最初发表在1918年出版的《庆祝麦克斯·普朗克60寿辰:德国物理学会演讲集》。1932年爱因斯坦将此文略加修改,作为普朗克文集《科学往何处去?》的序言。} \\

在科学的殿堂里有许多楼阁,住在里面的人真是各式各样,而引导他们到那里去的动机也各不相同。有许多人热爱科学是因为科学给他们一超乎常人的智力上的快感,科学是他们自己的特殊娱乐,他们在这种娱乐中下去寻求生动活泼的经验和对他们自己的雄心壮志的满足。在这座神殿里,另外还有才、许多人是为了纯粹的功利目的而把他们的脑力产物奉献到祭坛上的。如果上帝的一位天使跑来把所有属于这两种人赶出神殿,那么集结在那里的人就会大大减少,但是仍然会有一些人留在里面,我们的麦克斯·普朗克(量子力学奠基人之一,Max.Plank)就是其中之一。


我很明白在刚才想象中被轻易逐出的人里面也有很多卓越的人物,他们在建筑科学神殿中作出过很大也许是主要的贡献;许多情况下我们的天使也会觉得难以决定谁该不该被赶走。但有一点可以肯定,如果科研的殿堂里只由前两种人来构建的话,那么科学的殿堂也就顶多像是一个森林那样蔓延,里面除了各种爬藤,不会再有其它东西。因为对于这些人来说,只要碰上机会,任何人类活动的领域都是合适的;他们究竟成为工程师、官吏、商人还是科学家,完全取决于环境。现在我们再看那些在天使面前蒙恩的人。他们大多沉默寡言的、相当乖僻和孤独的人,但尽管有这些共同特点,他们之间却不象被赶走的一群那样彼此相似。究竟是什么力量把他们吸引到这座神殿中来的呢?这是一个难题,不能笼统地用一句话来回答,首先我同意叔本华所说的,他们来到科学殿堂的最强动机,是要逃避个人生活的苦难和悲催,进入到一个可以客观感知的世界。一个修养有素的人总是渴望逃避个人生活而进入客观知觉和思维的世界—这种愿望好比城市的人渴望逃避熙来攘往的环境,而到高山上享受幽寂的生活。


除了这种消极的动机外,还有一种积极的动机。这些人师徒构造一个最适合于他风范的、简约的、可以理解的画卷,然后他就试图用他的这种世界体系来代替经验的世界,并征服后者。这就是画家、诗人、思辨哲学家和自然科学家各按自己的方式去做的事。各人把世界体系及其构成作为他的感情生活的中枢,以便由此找到他在个人经验的狭小范围内所不能找到的宁静和安定。 

在所有可能的图像中,理论物理学家的世界图像占有什么地位呢?在描述各种关系时,它要求严密的精确性达到那种只有用数学语言才能达到的最高的标准。另一方面,物理学家必须极其严格地控制他的主题范围,必须满足于描述我们经验领域里的最简单事件。对于一切更为复杂的事件企图以理论物理学家所要求的精密性和逻辑上的完备性把它们重演出来,这就超出了人类理智所能及的范围。高度的纯粹性、明晰性和确定性要以完整性为代价。但是当人们胆小谨慎地把一切比较复杂而难以捉摸的东西都撇开不管时,那么能吸引我们去认识自然界的这一渺小部分的,究竟又是什么呢?难道这种谨小慎微的努力结果也够得上宇宙理论的美名吗? 

我认为,够得上的。因为,作为理论物理学结构基础的普遍定律,应当对任何自然现象都有效。有了它们,就有可能借助于单纯的演绎得出一切自然过程(包括生命过程)的描述,也就是它们的理论,只要这种演绎过程并不超出人类理智能力太多。因此,物理学家放弃他的世界体系的完整性,倒不是一个什么根本原则问题。 

物理学家的最高使命是得到那些普遍的基本定律,由此世界体系就能用单纯的演绎法建立起来。要通向这些定律,没有逻辑推理的途径,只有通过建立在经验的同感的理解之上的那种直觉。由于这种方法论上的不确定性,人们将认为这样就会有多种可能同样适用的理论物理学体系,这个看法在理论上无疑是正确的。但是物理学的发展表明,在某一时期里,在所有可想到的解释中,总有一个比其他的一些都高明得多。凡是真正深入研究过这一问题的人,都不会否认唯一决定理论体系的实际上是现象世界,尽管在现象和他们的理论原理之间并没有逻辑的桥梁;这就是莱布尼茨非常中肯地表述过的“先天的和谐”。物理学家往往责备研究认识论的人没有足够注意这个事实。我认为,几年前马赫和普朗克的论战,根源就在这里。

渴望看到这种先天的和谐,是无穷的毅力和耐心的源泉。我们看到,普朗克就是因此而专心致志于这门科学中的最普遍的问题,而不是使自己分心于比较愉快的和容易达到的目标上去的人。我常常听说,同事们试图把他的这种态度归因于非凡的意志和修养,但我认为这是错误的。促使人们去做这种工作的精神状态,是同宗教信奉者或谈恋爱的人的精神状态相类似的,他们每日的努力并非来自深思熟虑的意向或计划,而是直接来自激情。我们敬爱的普朗克今天就坐在这里,内心在笑我像孩子一样提着第欧根尼的风灯闹着玩。我们对他的爱戴不需要作老生常谈的说明,我们但愿他对科学的热爱将继续照亮他未来的道路,并引导他去解决今天理论物理学的最重要的问题。这问题是他自己提出来的,并且为了解决这问题他已经做了很多工作。祝他成功地把量子论同电动力学、力学统一于一个单一的逻辑体系里。



\end{document}